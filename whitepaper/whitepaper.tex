\documentclass[12pt]{article}
\title{A Decentralized Mnemonic Backup System}
\author{PriFi Labs Inc.}

\date{}

\usepackage{hyperref}
\usepackage{graphicx}

\usepackage{amsmath,proof,amsthm,amssymb}

\usepackage{setspace}
\usepackage{etoolbox}
\AtBeginEnvironment{quote}{\par\singlespacing\small}

\usepackage{enumitem}

\newcommand{\ms}[1]{\ensuremath{\mathsf{#1}}}
\newcommand{\msi}[1]{\ensuremath{\mathsfit{#1}}}

\newcommand{\tx}[3]{\ms{#1}[#2](#3)}
\newcommand{\qry}[3]{\ms{#1}(#3)}

\begin{document}

\maketitle

\begin{abstract}

\end{abstract}

\section{Introduction}

\begin{quote}
{\em ``If you do not own your private keys, you do not own your crypto''}
\end{quote}

This crypto adage encourages crypto enthusiasts to use non-custodial crypto wallets in which the private keys are directly stored on their device. However, using non-custodial wallets comes with the risk of losing all of our crypto assets if we lose our private keys. If our device gets lost, stolen or irremediably break down, our crypto assets might be locked forever \cite{LostPasswords}.\\

Fortunately, most non-custodial wallets such as Exodus (Bitcoin), Metamask (Ethereum) and Keplr (Secret Network) offer a way to recover our private keys by the mean of a ``recovery phrase'' also known as ``mnemonic seed phrase''. It is the 12-word phrase that we are being given the first time we setup our wallet. Here is an example of such a phrase: 

\begin{quote}
\begin{center}
{\tt witch fox practice feed shame open}\\
{\tt despair creek road again ice least}
\end{center}
\end{quote}

That phrase is important as it is used to generate the same private keys on demand (BIP39 standard \cite{BIP39}). We can use them as a backup or to import our wallet into a new device. Indeed, this mnemonic phrase is a really sensitive piece of information. Anyone who has access to this phrase would have full control over our crypto assets as explained in the [Metamask FAQ page]\cite{MetamaskFAQ}:

\begin{quote}
{\em ``MetaMask requires that you store your Secret Recovery Phrase in a safe place. It is the only way to recover your funds should your device crash or your browser reset. We recommend you to write it down. The most common method is to write your 12-word phrase on a piece of paper and store it safely in a place where only you have access. Note: if you lose your Secret Recovery Phrase, MetaMask can’t help you recover your wallet. Never give you Secret Recovery Phrase or your private key(s) to anyone or any site, unless you want them to have full control over your funds.''}\\
\end{quote}

As understood here, we are not supposed to remember that phrase like a password. Instead, it is recommended that we write it down on a piece of paper and store it in a ``safe place''. But where is that ``safe place'' exactly? In our physical wallet with our cash and credit cards? Not ideal as it could get lost or stolen. Inside our home? No ideal again as it could burn down. A safe deposit box in the vault of a bank? it might be the best solution after all but the logistics that comes with it makes that solution very cumbersome. \\
 
In the end, it is very inconvenient to store physical objects safely. So, could we design a simple application that would take custody of my passphrase and would allow me to recover by simply using my email? Yes we could build such an application, it would be something similar to a password manager such as {\em 1password} \cite{1password} for instance. However, such a solution requires 1) that the service provider is trustworthy an 2) that the whole application is secured. Indeed, such a centralized solution goes the philosophy of decentralized application. \\

In this paper, we propose a Decentralized Mnemonic Backup system that anyone can use to give custody to any blockchain passphrase to the Secret Network \cite{SecretNetwork} and recover it using a simple email. 

Iteration 1: user experience
Iteration 2: security hardening (Secret NFTs + key exchange protocol)
Iteration 3: improving reliability (Shamir' Secret Sharing sheme)

\section{Iteration 1: the user experience}

Once of the greatest feature of the Secret Contracts is that the data are encrypted. Therefore, we could implement ... 

\subsection{The Protocol}

\paragraph{Backup}

\begin{enumerate} 
\item {\bf Request} - Alice enters her $email$, a passphrase $pp$ and an $id$ for her passphrase on a webpage. After pressing the submit button, a script running inside the webpage sends a transaction {\bf $\ms{register}(email, id)$} to the Recovery Secret Contract. 
\item {\bf Registration} - The Secret Recovery Contract generates a random confirmation code and stores the record $(t_{id}, email, id, code)$ in the {\tt register} dataset. 
\item {\bf Emailing} - Once the request transaction has been validated, an event handler is triggered on the Mailer Backend that queries the Recovery Secret Contract for the email and confirmation code associated with the transaction id. Thus, the Mailer Backend sends the confirmation code to Alice by email. 
\item {\bf Confirmation} - Now Alice sees a webpage that asks her to enter the confirmation code sent by email. She opens her email and copies'n paste the confirmation code into her browser. After pressing the submit button, a script running inside the webpage sends a transaction $\ms{confirm}(email, id, pp, code)$ to the Recovery Secret Contract. 
\item {\bf Recording} - The Secret Recovery Contract verifies the verification and stores the record $(email, id, pp)$ in the {\tt confirm} dataset. 
\end{enumerate}

\paragraph{Recovery}

\begin{enumerate}
\item {\bf Request} - Alice enters her $email$ and an $id$ for her passphrase. After pressing the submit button, a script running inside the webpage sends a transaction {\bf $\ms{recover}(email, id)$} to the Recovery Secret Contract. 
\item {\bf Registration} - The Secret Recovery Contract generates a random verification code and stores the record $(t_{id}, email, id, code)$ in the {\tt recover} dataset. 
\item {\bf Emailing} - Once the request transaction has been validated, an event handler is triggered on the Mailer Backend that queries the Recovery Secret Contract for the email and verification code associated with the transaction id. Thus, the Mailer Backend sends the verification code to Alice by email. 
\item {\bf Verification} - Now, Alice sees a webpage that asks her to enter the verification code sent by email. She opens her email and copies'n paste that verification code into her browser. After pressing the submit button, a script running inside the page sends a query $\ms{verification}(email, id, code)$ to the Recovery Secret Contract. 
\item {\bf Response} - The Secret Recovery Contract verifies the verification code, retrieves the corresponding record from the {\tt recover} dataset and returns the passphrase.
\end{itemize}

\subsection{Security Analysis}

Although, this first iteration captures the user experience that we want, it is unfortunately not satisfactory in terms of security. There are three single points of failure: 1) Alice's account, 2) the Mailer Backend and 3) the Secret Recovery Contract. \\

What if Alice's account is hacked? As explained earlier, it is recommended that Alice uses a throwaway Secret account when registering a passphrase. As soon as the passphrase has been recorded, she can forget about this account since she won't need to recover her passphrase. However, what if the account's private key is leaked somehow? It would be bad since the attacker with this key could look into the transaction history and get the passphrase. \\

What if the Mailer Backend is hacked? Let's consider the two phases: backup and recovery. During the backup phase, the attacker could use the confirmation code to upload an arbitrary passphrase for Alice. This is a problem if Alice recovers what she believes is her original passphrase but is in fact the passphrase to an account that the attacker can access as well. Then, any new asset Alice puts in this account from now on can be stolen by the attacker. During recovery phase it is even worst since the attacker could use the verification code to query the Recovery Secret Contract directly and get the passphrase back. \\

What if the Recovery Secret Contract is hacked? This is by far the biggest threat here since the attacker will have access to all users' passphrases.  \\

Indeed, all of these security threats are not acceptable and we are going to address all of these issues by hardening our protocol in the next iteration. 

\section{Iteration  2: improving security}

She creates a throwaway account on the Secret Network and mints a Secret NFT embedding her passphrase as secret metadata. Then, she sends this NFT to her friend Bob. When the day comes that she needs to recover her passphrase, she can create yet another throwaway account and ask Bob to send her back her NFT top recover the passphrase. 

\subsection{Key Registration}

\paragraph{Step 1 - Registration Request}

\begin{enumerate}[label=1.\arabic*]
  \item Alice enters her $email$, her passphrase $m$, an $id$ for her passphrase, the number of shares $i$ she wants to generates and the threshold value $j$ and pressed the submit button
  \item Then, a script running inside the page generates an ECDH private and public key pair $(sec_A, pub_A)$ and a nonce $n_A$
  \item Finally, the script sends the transaction $\tx{request}{t_{id}, b_{id}}{email, id, pub_A, n_A}$ to the Secret Network using the throwaway Secret account that Alice has just provisioned.
\end{enumerate}

\paragraph{Step 2 - Registration Process}

\begin{enumerate}[label=2.\arabic*]
  \item The Secret Recovery Contract receives the $\ms{register}$ transaction request and generates a new ECDH private and public key pair $(sec_B, pub_B)$ and a nonce $n_B$
  \item Then, calculates the ECDH secret $s=\ms{ECDH}(sec_B, pub_B, pub_A)$ and derives the 128-bit AES symmetric key $k$ using the standard password-based key derivation function $\ms{PBFK2}$ and the concatenation of $n_A$ and $n_B$ as a salt $k=\ms{PBKDF2}(s, n_A || n_B)$
  \item Finally, the contract stores the record $(email, id, t_{id}, b_{id}, pub_B, n_B)$ in the {\tt register} dataset. 
\end{enumerate}

\paragraph{Step 3 - Confirmation Email}

\begin{enumerate}[label=3.\arabic*]
  \item Any transaction sent to the Secret Recovery Contract Contract triggers an event handler on the Mailer Backend. Once a transaction has been validated, the Mailer Backend sends a query $\qry{info}{t_id}$ to the Recovery Contract to get information about that transaction.
  \item The Secret Recovery Contract checks that the query comes from the Mailer Backend address and retrieves the record from the {\tt register} dataset and checks that the query has not expire based on the initial block id $b_{id}$ and the current block id on the Secret Network
  \item Then, it returns the $email$, the public key pair $pub_B$ and the nonce $n_B$.
  \item Finally, the mailer daemon sends an email to Alice with the confirmation code $(pub_B, n_B)$
\end{enumerate}

\paragraph{Step 4 - Confirmation Request}

\begin{enumerate}[label=4.\arabic*]
  \item Alice sees a webpage that asks her to enter the confirmation code sent by email. She opens her email and copy'n pastes that confirmation code into her browser. 
  \item Then, a script running inside the page calculates the ECDH secret $s=\ms{ECDH}(sec_A, pub_A, pub_B)$ and derives the 128-bit AES symmetric key $k$ using $\ms{PBFK2}$ and the concatenation of $n_A$ and $n_B$ as a salt $k=\ms{PBKDF2}(s, n_A || n_B)$
  \item The script calculates a hash-based message authentication code (abbreviated HMAC) using the key $k$ and the concatenation of the email and id $h=\ms{HMAC}(k, email || id)$
  \item Finally, the script sends the transaction $\tx{confirm}{t_{id}', b_{id}'}{email, id, h}$ to the Secret Network
\end{enumerate}

\paragraph{Step 5 - Confirmation Process}

\begin{enumerate}[label=5.\arabic*]
  \item The Secret Recovery Contract receives the $\ms{confirm}$ transaction and checks that the query has not expire based on the initial block id $b_{id}$ and the current block id on the Secret Network
  \item Then, it calculates the HMAC of the concatenation of the email and id using the key $h=\ms{HMAC}(k, email || id)$ and checks that this hash is strictly equal to the hash $h$ in the request
  \item Finally, the contract stores the tuple $(email, id, k)$ in the {\tt confirm} dataset.
\end{enumerate}

\paragraph{Step 6 - NFTs Minting}

\begin{enumerate}[label=6.\arabic*]
  \item Once the transaction has been validated, the client-side script can now encrypt (using AES in GCM mode) the passphrase $m$ with the key $k$ to obtain the ciphertext $c=E_{\ms{AES}}(k, m)$
  \item Then, the script concatenates the id and the ciphertext to generate the $i$ shares using the SSS algorithm: $(s_1,\cdots,s_i)=E_{\ms{SSS}}(id||c, i, j)$
  \item Finally, the script mints the $i$ Secret NFTs on the Secret Network
\end{enumerate}

Once the protocol is completed, Alice can now transfer the NFTs to her accointances. 

\subsection{Key Recovery}

\paragraph{Step 1 - Recovery Request}

\begin{enumerate}[label=1.\arabic*]
    \item Alice creates a new Secret Wallet and asks her friend to send the NFTs back. Once she has at least $j$ of those NFTs on her wallet, she goes to the recovery page that asks her to select the NFTs.
    \item Then, the script in the webpage collects the shares and extracts the $id$ and the ciphertext from the NFTs $id||c = D_{\ms{SSS}}(s_1,\cdots,s_j)$
    \item Finally, the script generates a nonce $n_A$ and sends the transaction $\tx{request}{t_{id}, b_{id}}{email, id, pub_A, n_A}$ to the Secret Network
\end{enumerate}

\paragraph{Step 2 - Recovery Process}

\begin{enumerate}[label=2.\arabic*]
    \item The Secret Recovery Contract receives the transaction and retrieves the record from the {\tt confirm} dataset using the email and id from the transaction
    \item Then, the script generates a nonce $n_B$ and stores the record $(email, id, t_{id}, b_{id}, n_B)$ in the {\tt register} dataset.
\end{enumerate}

\paragraph{Step 3 - Verification Email}

\begin{enumerate}[label=3.\arabic*]
  \item Any transaction sent to the Secret Recovery Contract Contract triggers an event handler on the Mailer Backend. Once a transaction has been validated, the Mailer Backend sends a query $\qry{info}{t_id}$ to the Recovery Contract to get information about that transaction.
  \item The Secret Recovery Contract checks that the query comes the Mailer Backend and retrieves the record from the {\tt register} dataset and checks that the query has not expire based on the initial block id $b_{id}$ and the current block id on the Secret Network
  \item Then, it returns the $email$, the public key the nonce $n_B$.
  \item Finally, the mailer daemon sends an email to Alice with the verification code $n_B$
\end{enumerate}

\paragraph{{\bf Step 4} - Verification Request}

\begin{enumerate}[label=4.\arabic*]
  \item Alice sees a webpage that asks her to enter the verification code sent by email. She opens her email and copy'n pastes that validation code into her browser.
  \item Then, the script calculates the hash of the concatenation of the two nonces $h=\ms{H_{\ms{SHA512}}}(n_A || n_B)$
  \item Finally, the script sends a query $\tx{verify}{t_{id}', b_{id}'}{email, id, h}$ to the Secret Network
\end{enumerate}

\paragraph{Step 5 - Recovery Process}

\begin{enumerate}[label=5.\arabic*]
    \item The Secret Recovery Contract receives the $\ms{verify}$ transaction and checks that the query has not expire based on the initial block id $b_{id}$ and the current block id on the Secret Network
  \item Then, it calculates the hash of the concatenation of the two nonces $h=\ms{H_{\ms{SHA512}}}(n_A || n_B)$ and checks that this hash is strictly equal to the hash $h$ in the request
  \item Finally, it returns the key $k$ back to the client
\end{enumerate}

\paragraph{Step 6 - Passphrase Decryption}

\begin{enumerate}[label=6.\arabic*]
  \item After receivving the key back, the client-side script can now decrypt the ciphertext $c$ to obtain the passphrase $m=D_{\ms{AES}}(k, c)$
\end{enumerate}

Once the protocol is completed, Alice sees her passphrase on the webpage.

\subsection{Usability Analysis}

This second iteration of our Mnemonic Backup System is more secured. If the Mailer Daemon is breached, the attacker will not be able to recover the key using the code without the user's nonce. If the Recovery Secret Contract is hacked, indeed the attacker will get all users keys. However, those keys are useless since the attacker cannot locate the NFTs to retrieve the encrypted passphrase.

\section{Iteration 3: improving reliability}

\subsection{The Protocol}

For the backup, Alice needs to specify the number of shares $i$ she wants to generates and the minimum number of shares $j$ needed to retrieve the passphrase (threshold value). Once, the encryption key has been fully registered in the contract, the script encrypts the passphrase, concatenates the id and the ciphertext to generate the $i$ shares using the SSS algorithm: $(s_1,\cdots,s_i)=E_{\ms{SSS}}(id||c, i, j)$. Finally, the script mints the $i$ Secret NFTs on the Secret Network. Once the protocol is completed, Alice can now transfer the NFTs to her accointances. 

For the recovery, Alice creates a new Secret Wallet and asks all of her friends to send their NFTs back. Once she has at least $j$ of those NFTs on her wallet, she goes to the recovery page that asks her to select the NFTs. Then, the script in the webpage collects the shares and extracts the $id$ and the ciphertext from the NFTs $id||c = D_{\ms{SSS}}(s_1,\cdots,s_j)$. Then, the protocol continues like in the previous iteration. 

\section{Value capture for Secret Network ecosystem}

This backup system will demonstrate that the Secret Network privacy model is adequate to keep sensitive information confidential. Key features of Secret NFTs: secret metada and private ownership \\ 

Why implementing our backup system on Secret Network rather than Ethereum? Is it because Ethereum-based NFTs do not have secret metadata? Not really since we could have created a password protected version of the share and hid it inside a public NFT image using Steganography \cite{Steganography}. The real problem is that if an attacker knows that an NFT contains a share to recover a mnemonic phrase, he or she will be able to locate all others NFTs that contains the other shares easily since ownership is public on Ethereum. The fact that Secret NFTs protects the ownership is the key feature that makes our backup system secured. 

\section{Conclusion and Future Work}

The key recovering system can be used outside of our Mnemonic Backup system. It can be used for more advanced cryptographic protocols that involve storing a secret key on the blockchain with the option of recovering it using an email. 

\begin{thebibliography}{9}

\bibitem{LostPasswords} Lost Passwords Lock Millionaires Out of Their Bitcoin Fortunes, \url{https://www.nytimes.com/2021/01/12/technology/bitcoin-passwords-wallets-fortunes.html}

\bibitem{BIP39} Mnemonic Generation (BIP39) Simply Explained, \url{https://medium.com/coinmonks/mnemonic-generation-bip39-simply-explained-e9ac18db9477}

\bibitem{MetamaskFAQ} Metamask FAQ page, \url{https://metamask.io/faqs/}

\bibitem{1password} 1password, \url{https://support.1password.com/forgot-account-password/}

\bibitem{SecretNetwork} Secret Network: A Privacy-Preserving Secret Contract \& Decentralized Application Platform, \url{https://scrt.network/graypaper}

\bibitem{Steganography} Steganography https://en.wikipedia.org/wiki/Steganography) 

\end{thebibliography}

\end{document}
